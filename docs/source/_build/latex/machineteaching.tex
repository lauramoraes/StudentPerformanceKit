%% Generated by Sphinx.
\def\sphinxdocclass{report}
\documentclass[letterpaper,10pt,english]{sphinxmanual}
\ifdefined\pdfpxdimen
   \let\sphinxpxdimen\pdfpxdimen\else\newdimen\sphinxpxdimen
\fi \sphinxpxdimen=.75bp\relax

\PassOptionsToPackage{warn}{textcomp}
\usepackage[utf8]{inputenc}
\ifdefined\DeclareUnicodeCharacter
% support both utf8 and utf8x syntaxes
  \ifdefined\DeclareUnicodeCharacterAsOptional
    \def\sphinxDUC#1{\DeclareUnicodeCharacter{"#1}}
  \else
    \let\sphinxDUC\DeclareUnicodeCharacter
  \fi
  \sphinxDUC{00A0}{\nobreakspace}
  \sphinxDUC{2500}{\sphinxunichar{2500}}
  \sphinxDUC{2502}{\sphinxunichar{2502}}
  \sphinxDUC{2514}{\sphinxunichar{2514}}
  \sphinxDUC{251C}{\sphinxunichar{251C}}
  \sphinxDUC{2572}{\textbackslash}
\fi
\usepackage{cmap}
\usepackage[T1]{fontenc}
\usepackage{amsmath,amssymb,amstext}
\usepackage{babel}



\usepackage{times}
\expandafter\ifx\csname T@LGR\endcsname\relax
\else
% LGR was declared as font encoding
  \substitutefont{LGR}{\rmdefault}{cmr}
  \substitutefont{LGR}{\sfdefault}{cmss}
  \substitutefont{LGR}{\ttdefault}{cmtt}
\fi
\expandafter\ifx\csname T@X2\endcsname\relax
  \expandafter\ifx\csname T@T2A\endcsname\relax
  \else
  % T2A was declared as font encoding
    \substitutefont{T2A}{\rmdefault}{cmr}
    \substitutefont{T2A}{\sfdefault}{cmss}
    \substitutefont{T2A}{\ttdefault}{cmtt}
  \fi
\else
% X2 was declared as font encoding
  \substitutefont{X2}{\rmdefault}{cmr}
  \substitutefont{X2}{\sfdefault}{cmss}
  \substitutefont{X2}{\ttdefault}{cmtt}
\fi


\usepackage[Bjarne]{fncychap}
\usepackage{sphinx}

\fvset{fontsize=\small}
\usepackage{geometry}

% Include hyperref last.
\usepackage{hyperref}
% Fix anchor placement for figures with captions.
\usepackage{hypcap}% it must be loaded after hyperref.
% Set up styles of URL: it should be placed after hyperref.
\urlstyle{same}
\addto\captionsenglish{\renewcommand{\contentsname}{Contents:}}

\usepackage{sphinxmessages}
\setcounter{tocdepth}{3}
\setcounter{secnumdepth}{3}


\title{Machine Teaching}
\date{Aug 01, 2019}
\release{}
\author{Laura de Oliveira F.\@{} Moraes}
\newcommand{\sphinxlogo}{\vbox{}}
\renewcommand{\releasename}{}
\makeindex
\begin{document}

\ifdefined\shorthandoff
  \ifnum\catcode`\=\string=\active\shorthandoff{=}\fi
  \ifnum\catcode`\"=\active\shorthandoff{"}\fi
\fi

\pagestyle{empty}
\sphinxmaketitle
\pagestyle{plain}
\sphinxtableofcontents
\pagestyle{normal}
\phantomsection\label{\detokenize{index::doc}}



\chapter{bkt module}
\label{\detokenize{bkt:module-bkt}}\label{\detokenize{bkt:bkt-module}}\label{\detokenize{bkt::doc}}\index{bkt (module)@\spxentry{bkt}\spxextra{module}}\index{BKT (class in bkt)@\spxentry{BKT}\spxextra{class in bkt}}

\begin{fulllineitems}
\phantomsection\label{\detokenize{bkt:bkt.BKT}}\pysiglinewithargsret{\sphinxbfcode{\sphinxupquote{class }}\sphinxcode{\sphinxupquote{bkt.}}\sphinxbfcode{\sphinxupquote{BKT}}}{\emph{hmm\_folder='hmm-scalable-818d905234a8600a8e3a65bb0f7aa4cf06423f1a'}, \emph{git\_commit='818d905234a8600a8e3a65bb0f7aa4cf06423f1a'}}{}
Bases: \sphinxcode{\sphinxupquote{object}}
\index{download() (bkt.BKT method)@\spxentry{download()}\spxextra{bkt.BKT method}}

\begin{fulllineitems}
\phantomsection\label{\detokenize{bkt:bkt.BKT.download}}\pysiglinewithargsret{\sphinxbfcode{\sphinxupquote{download}}}{}{}
This implementation is a wrapper around the
HMM-scalable tool ( \sphinxurl{http://yudelson.info/hmm-scalable}).
This function will download and install the original implementation.
\begin{quote}\begin{description}
\item[{Returns}] \leavevmode
\sphinxstylestrong{self}

\item[{Return type}] \leavevmode
object

\end{description}\end{quote}
\subsubsection*{Notes}

This is a wrapper around the HMM-scalable tool (\sphinxurl{http://yudelson.info/hmm-scalable}).

\end{fulllineitems}

\index{fit() (bkt.BKT method)@\spxentry{fit()}\spxextra{bkt.BKT method}}

\begin{fulllineitems}
\phantomsection\label{\detokenize{bkt:bkt.BKT.fit}}\pysiglinewithargsret{\sphinxbfcode{\sphinxupquote{fit}}}{\emph{data}, \emph{q\_matrix}, \emph{solver='bw'}, \emph{iterations=200}}{}
Fit BKT model to data.
As of July 2019, just default parameters are allowed.
\begin{quote}\begin{description}
\item[{Parameters}] \leavevmode\begin{itemize}
\item {} 
\sphinxstyleliteralstrong{\sphinxupquote{data}} (\sphinxstyleliteralemphasis{\sphinxupquote{\{array-like\}}}\sphinxstyleliteralemphasis{\sphinxupquote{, }}\sphinxstyleliteralemphasis{\sphinxupquote{shape}}\sphinxstyleliteralemphasis{\sphinxupquote{ (}}\sphinxstyleliteralemphasis{\sphinxupquote{n\_steps}}\sphinxstyleliteralemphasis{\sphinxupquote{, }}\sphinxstyleliteralemphasis{\sphinxupquote{3}}\sphinxstyleliteralemphasis{\sphinxupquote{)}}) \textendash{} Sequence of students steps. Each of the three dimensions are:
Observed outcome: 0 for fail and 1 for success
Student id: student unique identifier
Question id: question id in q\_matrix

\item {} 
\sphinxstyleliteralstrong{\sphinxupquote{q\_matrix}} (\sphinxstyleliteralemphasis{\sphinxupquote{matrix}}\sphinxstyleliteralemphasis{\sphinxupquote{, }}\sphinxstyleliteralemphasis{\sphinxupquote{shape}}\sphinxstyleliteralemphasis{\sphinxupquote{ (}}\sphinxstyleliteralemphasis{\sphinxupquote{n\_questions}}\sphinxstyleliteralemphasis{\sphinxupquote{, }}\sphinxstyleliteralemphasis{\sphinxupquote{n\_concepts}}\sphinxstyleliteralemphasis{\sphinxupquote{)}}) \textendash{} Each row is a question and each column a concept.
If the concept is present in the question, the
correspondent cell should contain 1, otherwise, 0.

\item {} 
\sphinxstyleliteralstrong{\sphinxupquote{solver}} (\sphinxstyleliteralemphasis{\sphinxupquote{string}}\sphinxstyleliteralemphasis{\sphinxupquote{, }}\sphinxstyleliteralemphasis{\sphinxupquote{optional}}) \textendash{} Algorithm used to fit the BKT model. Available solvers are:
‘bw’: Baum-Welch (default)
‘gd’: Gradient Descent
‘cgd\_pr’: Conjugate Gradient Descent (Polak-Ribiere)
‘cgd\_fr’: Conjugate Gradient Descent (Fletcher\textendash{}Reeves)
‘cgd\_hs’: Conjugate Gradient Descent (Hestenes-Stiefel)

\item {} 
\sphinxstyleliteralstrong{\sphinxupquote{iterations}} (\sphinxstyleliteralemphasis{\sphinxupquote{integer}}\sphinxstyleliteralemphasis{\sphinxupquote{, }}\sphinxstyleliteralemphasis{\sphinxupquote{optional}}) \textendash{} Maximum number of iterations

\end{itemize}

\item[{Returns}] \leavevmode
\sphinxstylestrong{self}

\item[{Return type}] \leavevmode
object

\end{description}\end{quote}
\subsubsection*{Notes}

This is a wrapper around the HMM-scalable tool (\sphinxurl{http://yudelson.info/hmm-scalable})

\end{fulllineitems}

\index{get\_params() (bkt.BKT method)@\spxentry{get\_params()}\spxextra{bkt.BKT method}}

\begin{fulllineitems}
\phantomsection\label{\detokenize{bkt:bkt.BKT.get_params}}\pysiglinewithargsret{\sphinxbfcode{\sphinxupquote{get\_params}}}{}{}
Get fitted params.
\begin{quote}\begin{description}
\item[{Returns}] \leavevmode
\sphinxstylestrong{params}

\item[{Return type}] \leavevmode
list. List containing the prior, transition and emission values for each skill.

\end{description}\end{quote}

\end{fulllineitems}

\index{predict() (bkt.BKT method)@\spxentry{predict()}\spxextra{bkt.BKT method}}

\begin{fulllineitems}
\phantomsection\label{\detokenize{bkt:bkt.BKT.predict}}\pysiglinewithargsret{\sphinxbfcode{\sphinxupquote{predict}}}{\emph{data}, \emph{q\_matrix}, \emph{model\_file=None}}{}
Predict student outcomes based on trained model. This is just the hard-assigment
(highest probability) of the outcome probabilities.
\begin{quote}\begin{description}
\item[{Parameters}] \leavevmode
\sphinxstyleliteralstrong{\sphinxupquote{data}} (\sphinxstyleliteralemphasis{\sphinxupquote{\{array-like\}}}\sphinxstyleliteralemphasis{\sphinxupquote{, }}\sphinxstyleliteralemphasis{\sphinxupquote{shape}}\sphinxstyleliteralemphasis{\sphinxupquote{ (}}\sphinxstyleliteralemphasis{\sphinxupquote{n\_steps}}\sphinxstyleliteralemphasis{\sphinxupquote{, }}\sphinxstyleliteralemphasis{\sphinxupquote{3}}\sphinxstyleliteralemphasis{\sphinxupquote{)}}) \textendash{} Sequence of students steps. Each of the three dimensions are:
Observed outcome: 0 for fail and 1 for success
Student id: student unique identifier
Question id: question id in q\_matrix

\item[{Returns}] \leavevmode
\sphinxstylestrong{outcome} \textendash{} Outcomes for steps in data. Outcome 0 is incorrect and outcome 1 is correct.

\item[{Return type}] \leavevmode
\{array-like\}, shape (n\_steps,)

\end{description}\end{quote}
\subsubsection*{Notes}

This is a wrapper around the HMM-scalable tool (\sphinxurl{http://yudelson.info/hmm-scalable})

\end{fulllineitems}

\index{predict\_proba() (bkt.BKT method)@\spxentry{predict\_proba()}\spxextra{bkt.BKT method}}

\begin{fulllineitems}
\phantomsection\label{\detokenize{bkt:bkt.BKT.predict_proba}}\pysiglinewithargsret{\sphinxbfcode{\sphinxupquote{predict\_proba}}}{\emph{data}, \emph{q\_matrix}, \emph{model\_file=None}}{}
Predict student outcome probabilities based on trained model.
\begin{quote}\begin{description}
\item[{Parameters}] \leavevmode
\sphinxstyleliteralstrong{\sphinxupquote{data}} (\sphinxstyleliteralemphasis{\sphinxupquote{\{array-like\}}}\sphinxstyleliteralemphasis{\sphinxupquote{, }}\sphinxstyleliteralemphasis{\sphinxupquote{shape}}\sphinxstyleliteralemphasis{\sphinxupquote{ (}}\sphinxstyleliteralemphasis{\sphinxupquote{n\_steps}}\sphinxstyleliteralemphasis{\sphinxupquote{, }}\sphinxstyleliteralemphasis{\sphinxupquote{3}}\sphinxstyleliteralemphasis{\sphinxupquote{)}}) \textendash{} Sequence of students steps. Each of the three dimensions are:
Observed outcome: 0 for fail and 1 for success
Student id: student unique identifier
Question id: question id in q\_matrix

\item[{Returns}] \leavevmode
\sphinxstylestrong{outcome} \textendash{} Outcome probabilites for steps in data. Column 0 corresponds to outcome 0 (incorrect)
and column 1 to outcome 1 (correct)

\item[{Return type}] \leavevmode
\{array-like\}, shape (n\_steps, 2)

\end{description}\end{quote}
\subsubsection*{Notes}

This is a wrapper around the HMM-scalable tool (\sphinxurl{http://yudelson.info/hmm-scalable})

\end{fulllineitems}

\index{set\_params() (bkt.BKT method)@\spxentry{set\_params()}\spxextra{bkt.BKT method}}

\begin{fulllineitems}
\phantomsection\label{\detokenize{bkt:bkt.BKT.set_params}}\pysiglinewithargsret{\sphinxbfcode{\sphinxupquote{set\_params}}}{\emph{params}}{}
Set model params. No validation is done for this function.
Make sure the params variable is in the expected format.
\begin{quote}\begin{description}
\item[{Returns}] \leavevmode
\sphinxstylestrong{self}

\item[{Return type}] \leavevmode
object

\end{description}\end{quote}

\end{fulllineitems}


\end{fulllineitems}



\chapter{simulate\_student module}
\label{\detokenize{simulate_student:module-simulate_student}}\label{\detokenize{simulate_student:simulate-student-module}}\label{\detokenize{simulate_student::doc}}\index{simulate\_student (module)@\spxentry{simulate\_student}\spxextra{module}}\index{SimulateStudent (class in simulate\_student)@\spxentry{SimulateStudent}\spxextra{class in simulate\_student}}

\begin{fulllineitems}
\phantomsection\label{\detokenize{simulate_student:simulate_student.SimulateStudent}}\pysiglinewithargsret{\sphinxbfcode{\sphinxupquote{class }}\sphinxcode{\sphinxupquote{simulate\_student.}}\sphinxbfcode{\sphinxupquote{SimulateStudent}}}{\emph{pi}, \emph{A}, \emph{B}}{}
Bases: \sphinxcode{\sphinxupquote{object}}
\index{random\_MN\_draw() (simulate\_student.SimulateStudent method)@\spxentry{random\_MN\_draw()}\spxextra{simulate\_student.SimulateStudent method}}

\begin{fulllineitems}
\phantomsection\label{\detokenize{simulate_student:simulate_student.SimulateStudent.random_MN_draw}}\pysiglinewithargsret{\sphinxbfcode{\sphinxupquote{random\_MN\_draw}}}{\emph{n}, \emph{probs}}{}
get X random draws from the multinomial distribution whose probability is given by ‘probs’

\end{fulllineitems}

\index{simulate() (simulate\_student.SimulateStudent method)@\spxentry{simulate()}\spxextra{simulate\_student.SimulateStudent method}}

\begin{fulllineitems}
\phantomsection\label{\detokenize{simulate_student:simulate_student.SimulateStudent.simulate}}\pysiglinewithargsret{\sphinxbfcode{\sphinxupquote{simulate}}}{\emph{nSteps}}{}
given an HMM = (A, B, pi), simulate state and observation sequences

\end{fulllineitems}


\end{fulllineitems}



\chapter{Indices and tables}
\label{\detokenize{index:indices-and-tables}}\begin{itemize}
\item {} 
\DUrole{xref,std,std-ref}{genindex}

\item {} 
\DUrole{xref,std,std-ref}{modindex}

\item {} 
\DUrole{xref,std,std-ref}{search}

\end{itemize}


\renewcommand{\indexname}{Python Module Index}
\begin{sphinxtheindex}
\let\bigletter\sphinxstyleindexlettergroup
\bigletter{b}
\item\relax\sphinxstyleindexentry{bkt}\sphinxstyleindexpageref{bkt:\detokenize{module-bkt}}
\indexspace
\bigletter{s}
\item\relax\sphinxstyleindexentry{simulate\_student}\sphinxstyleindexpageref{simulate_student:\detokenize{module-simulate_student}}
\end{sphinxtheindex}

\renewcommand{\indexname}{Index}
\printindex
\end{document}